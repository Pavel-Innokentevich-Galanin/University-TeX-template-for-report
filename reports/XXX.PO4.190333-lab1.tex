\documentclass[12pt, a4paper, simple]{eskdtext}

\usepackage{config/main.env}
\usepackage{config/report.env}
\usepackage{styles/listing}
\usepackage{styles/lists}
\usepackage{styles/SectionMargins}
\usepackage{styles/table}
\usepackage{styles/TableOfContent}
\usepackage{styles/url}

\begin{document}
  \begin{ESKDtitlePage}
  \ESKDstyle{empty}
  \begin{center}
    \envReportMinistr \\
    \envReportEducation \\
    \envReportUniversity \\
    \envReportCathedra \\
  \end{center}

  \vfill

  \begin{center}
    Тема: <<\envReportTitle>>
  \end{center}

  \vfill

  \begin{center}
    Отчёт лабораторной работы №\envReportLabNumber \\
    по дисциплине \envReportSubject \\
  \end{center}

  \vfill

  \begin{flushright}
    \begin{minipage}[t]{7cm}
      Выполнил: \\
      \envReportStudentInfo \\
      \hspace{0pt} \\
      Проверил: \\
      \envReportTeacherInfo \\
    \end{minipage}
  \end{flushright}

  \vfill

  \begin{center}
    \envReportCity~\ESKDtheYear
  \end{center}
\end{ESKDtitlePage}


  % = = = = = = = =

  \ESKDstyle{empty}
  \begin{center}
    \textbf{Отчёт лабораторной работы №\envReportLabNumber}
  \end{center}

  % = = = = = = = =

  \paragraph{} \textbf{Тема}: <<\envReportTitle>>

  % = = = = = = = =

  \paragraph{} \textbf{Цель}:
  Знакомство с \TeX.

  % = = = = = = = =

  \paragraph{} \textbf{Что нужно сделать}:

  Получить из \TeX~файлов PDF файл.

  % = = = = = = = =

  \begin{center}
    \textbf{Ход работы}
  \end{center}

  % = = = = = = = =

  Устанавливаем Git, Docker, docker-compose, make на Ubuntu 22.10.

  \begin{lstlisting}[
    language=bash,
    frame=single,
    rulecolor=\color{blue},
    name={Терминал},
  ]
sudo apt update
sudo apt install -y git
sudo apt install -y docker.io docker-compose
sudo apt install -y make
\end{lstlisting}

  Чтобы выполнять команды нужно писать sudo docker, sudo docker-compose.
  Чтобы не писать sudo перед docker и docker-compose, то добавим пользователя в группу \cite{docker-without-sudo}.

  \begin{lstlisting}[
    language=bash,
    frame=single,
    rulecolor=\color{blue},
    name={Терминал},
  ]
sudo apt update
sudo apt install -y git
sudo apt install -y docker.io docker-compose
sudo apt install -y make

sudo groupadd docker
sudo gpasswd -a $USER docker 
newgrp docker
\end{lstlisting}

  Скачиваем репозиторий из GitHub \cite{github}.

  \begin{lstlisting}[
    language=bash,
    frame=single,
    rulecolor=\color{blue},
    name={Терминал},
  ]
ssh-keygen # Enter, Enter, Enter
cat ~/.ssh/id_rsa.pub # Copy and paste key to https://github.com/settings/ssh/new
git clone git@github.com:Pavel-Innokentevich-Galanin/University-TeX-template-for-report.git

cd University-TeX-template-for-report
\end{lstlisting}

  Запускаем сборку

  \begin{lstlisting}[
    language=bash,
    frame=single,
    rulecolor=\color{blue},
    name={Терминал},
  ]
make
\end{lstlisting}

  % = = = = = = = =

  \newpage

  \lstinputlisting[
    language=tex,
    frame=single,
    rulecolor=\color{magenta},
  ]{config/main.env.sty}

  \lstinputlisting[
    language=tex,
    frame=single,
    rulecolor=\color{magenta},
  ]{config/report.env.sty}

  % = = = = = = = =

  % Крутая штука изображена на рис.~\ref{fig:kruto}.

  % \begin{figure}[!h]
  %   \centering
  %   \includegraphics[width=18cm]
  %   {images/picture1.png}
  %   \caption{\_}
  %   \label{fig:kruto}
  % \end{figure}

  % = = = = = = = =

  % Крутая штука (см. рисунок~\ref{fig:mega}) и крутая штука (см. рисунок~\ref{fig:linux})

  % \begin{figure}[!h]
  %   \centering

  %   \begin{minipage}{0.49\textwidth}
  %     \centering

  %     \includegraphics[height=5cm]
  %     {images/picture1.png}

  %     \caption{\_}

  %     \label{fig:mega}
  %   \end{minipage}
  %   \begin{minipage}{0.49\textwidth}
  %     \centering

  %     \includegraphics[height=5cm]
  %     {images/picture2.png}

  %     \caption{\_}

  %     \label{fig:linux}
  %   \end{minipage}
  % \end{figure}

  % = = = = = = = =

  \newpage
  \begingroup
    \phantomsection
    \addcontentsline{toc}{section}{СПИСОК ИСПОЛЬЗОВАННЫХ ИСТОЧНИКОВ}
    \section*{Список использованных источников} %\section*{СПИСОК ИСПОЛЬЗОВАННЫХ ИСТОЧНИКОВ}

    \renewcommand{\addcontentsline}[3]{}% Remove functionality of \addcontentsline
    \renewcommand{\section}[2]{}% Remove functionality of \section

    \begin{thebibliography}{}

      \bibitem{docker-without-sudo}
      Как использовать Docker без sudo на Ubuntu
      [Электронный ресурс]. -
      Режим доступа:
      \url{https://itsecforu.ru/2018/04/12/%D0%BA%D0%B0%D0%BA-%D0%B8%D1%81%D0%BF%D0%BE%D0%BB%D1%8C%D0%B7%D0%BE%D0%B2%D0%B0%D1%82%D1%8C-docker-%D0%B1%D0%B5%D0%B7-sudo-%D0%BD%D0%B0-ubuntu}.
      Дата доступа: 08.03.2023.

      \bibitem{github}
      GitHub
      [Electronic resource]. -
      Mode of access:
      \url{https://github.com}.
      Date of access: 08.03.2023.
    \end{thebibliography}
  \endgroup
  % = = = = = = = = = = = = = = = =
\end{document}
